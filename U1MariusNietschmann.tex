\documentclass[11pt]{article}
\usepackage[T1]{fontenc}
\usepackage[german]{babel}
\usepackage[utf8]{inputenc}

%% margins
\usepackage{geometry}
\geometry{
  left=2.5cm,
  right=2.5cm,
  top=2.5cm,
  bottom=2.5cm
}

%% ams/thm
\usepackage{amsmath, amssymb, amsthm, mathtools, cancel, graphicx, wrapfig}

%%\begin{thm config/technicalities}
\theoremstyle{plain}
\newtheorem{thm}{Satz}[section]
\newtheorem{lem}[thm]{Lemma}
\newtheorem*{prop}{Proposition}
\newtheorem*{cor}{Korollar}
\newtheorem*{beh}{Behauptung}

\theoremstyle{definition}
\newtheorem{defn}[thm]{Definition}
\newtheorem{exmp}[thm]{Beispiel}

\theoremstyle{remark}
\newtheorem*{rem}{Bemerkung}

\usepackage{thmtools}

\declaretheoremstyle[%
  spaceabove=1.5ex,%
  spacebelow=6pt,%
  headfont=\normalfont\itshape,%
  postheadspace=1em,%
  qed=\qedsymbol%
]{mystyle} 
\declaretheorem[name={Beweis},style=mystyle,unnumbered,
]{prf}

\renewcommand{\proof}{\prf}

%% paragraph spacing/indentation
\usepackage{parskip}
\usepackage{setspace}
\setlength{\parindent}{0cm}

\makeatletter
\def\thm@space@setup{%
  \thm@preskip=\parskip \thm@postskip=0pt
}
\makeatother
%\end{}

%% plotting
%\usepackage{pgfplots}
%\usepackage{tikz}

\usepackage{enumitem}
%\setitemize{noitemsep,topsep=0pt,parsep=0pt,partopsep=0pt}

%% lightning
\usepackage{marvosym}

%% widecheck 
%\usepackage{mathabx}

%% sans-serif font
%\renewcommand{\familydefault}{\sfdefault}

% shortcuts
\newcommand{\N}{\mathbb{N}}
\newcommand{\Z}{\mathbb{Z}}
\newcommand{\Q}{\mathbb{Q}}
\newcommand{\R}{\mathbb{R}}
\newcommand{\C}{\mathbb{C}}
\newcommand{\K}{\mathbb{K}}
\newcommand{\Sn}{\mathbb{S}_n}
\renewcommand{\S}{\mathcal{S}}

\newcommand{\D}{\displaystyle}

\newcommand{\ep}{\varepsilon}
\newcommand{\ph}{\varphi}
\newcommand{\longto}{\longrightarrow}
\DeclareMathOperator{\kgV}{\mathrm{kgV}}
\DeclareMathOperator{\ggT}{\mathrm{ggT}}
\DeclareMathOperator{\ord}{\mathrm{ord}\,}
\DeclareMathOperator{\sgn}{\mathrm{sgn}\,}
\newcommand{\id}{\mathrm{id}}

\DeclareMathOperator{\Fou}{\mathcal{F}\!}

% fancy head/foot
\usepackage{fancyhdr}
\fancyhead[R]{M. Nietschmann}
\fancyhead[L]{}


\begin{document}
\pagestyle{fancy}
\thispagestyle{plain}

\rule{\textwidth}{.5pt}
\begin{center}
\Huge{Theoretische Physik - Übungsblatt 1}
\end{center}

\rule{\textwidth}{.5pt}
\text{} \hfill M. Nietschmann 


%%%%%%%%%%%%%%%%%%%%%%%%%%%%%%%
%%							 %%
%%   !!!Hier geht's los!!!   %%
%%							 %%
%%%%%%%%%%%%%%%%%%%%%%%%%%%%%%%


\section{Kurven und Flächen im $ \R^3 $} 

Es sind $ \vec{x},\vec{r} \in \R^3 $. 
\begin{itemize}
\item[a)] 
$ | \vec{x} | = 1 $ beschreibt die Einheitssphäre des $ \R^3 $. 

\item[b)] 
$ | \vec{x} - \vec{x}_0 | = R $ mit festem $ R > 0 $ sowie $ \vec{x}_0 \in \R^3 $ beschreibt eine Kugel des Radius $ R $ mit Mittelpunkt $ \vec{x}_0 $. 

\item[c)] 
Die Gleichung $ \vec{x} \cdot \vec{e} = 0 $ mit festem $ \vec{e} \in \R^3 $ 
wird von allen Vektoren des orthogonalen Komplementes des Erzeugnisses von $ \vec{e} $ erfüllt, d.h. $ \vec{x} \in \operatorname{span} \{ \vec{e} \}^\perp $. 
Das sind alle Vektoren des $ \R^3 $, falls $ \vec{e} $ der Nullvektor ist, andernfalls liegen alle diese Vektoren in einer Hyperebene, in diesem (dreidimensionalen) Falle also eine Ebene durch den Koordinatenursprung. 

\item[d)] $\langle r, k \rangle = \Vert k \Vert_2^2$ für festes $k \in \R^3$.

\item[e)] $x \times a = b \times a$ mit festen $a, b \in \R^3$.

\item[f)] $(\ph \cos\ph, \sin\ph, C\ph)$ mit $\ph \in \R$ und festem $C \in \R$.

\item[g)] $(r\sin(b) \cos(l), r \sin(b) \sin(l), r \cos(b))$ mit $b \in [0, \frac{\pi}{2}], \, l \in [0, 2\pi)$.

\end{itemize}


\section{Gradient in krummlinigen Koordinaten}

\begin{itemize}
\item[a)] 
$ \D \nabla \phi_P (r) = -\frac{1}{r^2} $ falls $ r \neq 0 $. Für $ r=0 $ ist $ \phi_P $ nicht differenzierbar. 

\item[b)] 
$ \D \nabla \phi_D (r,\vartheta) = \left( -2 \frac{\cos \vartheta}{r^3} , - \frac{\sin \vartheta}{r^2} \right) = -r^{-3} (2\cos \vartheta,r\sin\vartheta) $ falls $ r \neq 0 $. Für $ r=0 $ ist $ \phi_D $ nicht differenzierbar. 
\end{itemize}


\section{Divergenz}

\begin{itemize}
\item[a)] 
$ \operatorname{div} \vec{F} 
= \operatorname{div} 
	\begin{pmatrix} Kxyz \\ Kxyz \\ Kxyz \end{pmatrix} 
= K \begin{pmatrix} yz \\ xz \\ xy \end{pmatrix}. $ 

\item[b)] 

\item[c)] 
$ \operatorname{div} \vec{v} (x,y,z) 
= \operatorname{div} \sqrt{x^2+y^2+z^2} \vec{a} 
= \frac{x+y+z}{\sqrt{x^2+y^2+z^2}} \vec{a} 
= \frac{\| \vec{v} \|_1}{\| \vec{v} \|_2} \vec{a}. $ 

\item[d)] 

\item[e)]  

\end{itemize}



\end{document}
