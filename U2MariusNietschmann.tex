\documentclass[11pt]{article}
\usepackage[T1]{fontenc}
\usepackage[german]{babel}
\usepackage[utf8]{inputenc}

%% margins
\usepackage{geometry}
\geometry{
  left=2.5cm,
  right=2.5cm,
  top=2.5cm,
  bottom=2.5cm
}

%% ams/thm
\usepackage{amsmath, amssymb, amsthm, mathtools, cancel}

%%\begin{thm config/technicalities}
\theoremstyle{plain}
\newtheorem{thm}{Satz}[section]
\newtheorem{lem}[thm]{Lemma}
\newtheorem*{prop}{Proposition}
\newtheorem*{cor}{Korollar}
\newtheorem*{beh}{Behauptung}

\theoremstyle{definition}
\newtheorem{defn}[thm]{Definition}
\newtheorem{exmp}[thm]{Beispiel}
\newtheorem{rem}[thm]{Bemerkung}

%% paragraph spacing/indentation
\usepackage{parskip}
\usepackage{setspace}
\setlength{\parindent}{0cm}

\makeatletter
\def\thm@space@setup{%
  \thm@preskip=\parskip \thm@postskip=0pt
}
\makeatother
%\end{}

%% plotting
%\usepackage{pgfplots}
%\usepackage{tikz}

\usepackage{enumitem}
%\setitemize{noitemsep,topsep=0pt,parsep=0pt,partopsep=0pt}

%% lightning
\usepackage{marvosym}

%% sans-serif font
%\renewcommand{\familydefault}{\sfdefault}

% shortcuts
\newcommand{\N}{\mathbb{N}}
\newcommand{\Z}{\mathbb{Z}}
\newcommand{\Q}{\mathbb{Q}}
\newcommand{\R}{\mathbb{R}}
\newcommand{\C}{\mathbb{C}}
\newcommand{\K}{\mathbb{K}}

\newcommand{\D}{\displaystyle}

\newcommand{\ep}{\varepsilon}
\newcommand{\ph}{\varphi}
\newcommand{\longto}{\longrightarrow}
\newcommand{\kgV}{\mathrm{kgV}}
\newcommand{\ggT}{\mathrm{ggT}}
\newcommand{\ord}{\mathrm{ord}\,}
\newcommand{\sgn}{\mathrm{sgn}\,}
\newcommand{\id}{\mathrm{id}}
%\newcommand{\div}{\operatorname{div}}
\newcommand{\rot}{\operatorname{rot}}

% fancy head/foot
\usepackage{fancyhdr}
\fancyhead[R]{M. Nietschmann, M. Böhl}
\fancyhead[L]{}

\begin{document}
\pagestyle{fancy}
\thispagestyle{plain}

\rule{\textwidth}{.5pt}
\begin{center}
\Huge{Theoretische Physik - Übungsblatt 2}
\end{center}

\rule{\textwidth}{.5pt}
\text{} \hfill A. Kanz, R. Müller, M. Nietschmann, M. Böhl



\section{Flussintegral bei radialer Strömung}


\section{Kurvenintegral bei Scherströmung}
Gegeben ist das Vektorfeld $v(r) = (0, x, 0)$ für $r = (x,y,z) \in \R^3$. Sei nun $k \in (0, \infty)$ und $Q(k) = \{ (x,y,0) \in \R^3 \mid \max \{ x,y \} \leq \frac{k}{2} \}$ das Quadrat mit Seitenlänge $k$ und dem Koordinatenursprung als Mittelpunkt. Um den den Rand des Quadrats zu parametrisieren, definieren wir zunächst $\gamma_i : [0,1] \longto \R^3$ durch
\begin{align*}
\gamma_1 (t) &= (1, 2t-1, 0)\\
\gamma_2 (t) &= (1-2t, 1, 0)\\
\gamma_3 (t) &= (-1, 1-2t, 0)\\
\gamma_4 (t) &= (2t-1, -1, 0).
\end{align*}

Dann parametrisiert
\[ \gamma_k: [0,4] \longto \R^3,\; t \longmapsto \begin{cases}
\tfrac{k}{2}\gamma_1(t) &, t \in [0,1)\\
\tfrac{k}{2}\gamma_2(t-1) &, t \in [1,2)\\
\tfrac{k}{2}\gamma_3(t-2) &, t \in [2,3)\\
\tfrac{k}{2}\gamma_4(t-3) &, t \in [3,4]
\end{cases} \]

den Rand des Quadrates $Q(k)$. Nun können wir das Kurvenintegral berechnen:
\begin{align*}
\oint_{\gamma_k} v(r)\cdot dr &= \int_0^4 \langle v(\gamma_k(t)), \dot \gamma_k (t) \rangle dt \\
&= \frac{k^2}{4} \left( \int_0^1 \langle (0,1,0), (0,2,0) \rangle dt + \int_0^1 \langle (0, 1-2t, 0), (-2,0,0) \rangle dt \right.\\
&\quad \left. + \int_0^1 \langle (0,-1,0), (0,-2,0) \rangle dt + \int_0^1 \langle (0, 2t-1, 0), (2,0,0) \rangle dt \right)\\
&= 2\cdot\frac{k^2}{4} \int_0^1 2 dt = k^2.
\end{align*}

Die Rotation berechnet sich als $\rot v(r) = (0,0,1)$.


\section{Kurvenintegral eines azimutalen Geschwindigkeitsfeldes}
Gegeben ist das Vektorfelt $v(r) = (-y, x, 0)$ für $r = (x,y,z) \in \R^3$. Sei $R \in (0, \infty)$ und $K(R)$ der konzentrische Kreis mit Radius $R$ in der Ebene $z = 0$. Der Rand von $K(R)$ wird von der Kurve
\[ \gamma_R : [0, 2\pi) \longto \R^3, \; \ph \longmapsto R (\cos\ph, \sin\ph, 0) \]
``gegen den Urzeigersinn umfahren''. Es gilt
\[ \oint_{\gamma_R} v(r) \cdot dr = R^2 \int_0^{2\pi} (-\sin\ph, \cos\ph, 0) \cdot (-\sin\ph, \cos\ph, 0) d\ph = R^2 \int_0^{2\pi} \sin^2 \ph + \cos^2 \ph d\ph = 2\pi R^2 \]
und $\rot v(r) = (0,0,2)$.






\end{document}
